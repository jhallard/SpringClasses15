
% Programming/Coding Assignment
% LaTeX Template
%
% This template has been downloaded from:
% http://www.latextemplates.com
%
% Original author:
% Ted Pavlic (http://www.tedpavlic.com)
%
% Note:
% The \lipsum[#] commands throughout this template generate dummy text
% to fill the template out. These commands should all be removed when 
% writing assignment content.
%
% This template uses a Perl script as an example snippet of code, most other
% languages are also usable. Configure them in the "CODE INCLUSION 
% CONFIGURATION" section.
%
%%%%%%%%%%%%%%%%%%%%%%%%%%%%%%%%%%%%%%%%%

%----------------------------------------------------------------------------------------
%   PACKAGES AND OTHER DOCUMENT CONFIGURATIONS
%----------------------------------------------------------------------------------------

\documentclass{article}

\usepackage{fancyhdr} % Required for custom headers
\usepackage{lastpage} % Required to determine the last page for the footer
\usepackage{extramarks} % Required for headers and footers
\usepackage[usenames,dvipsnames]{color} % Required for custom colors
\usepackage{graphicx} % Required to insert images
\usepackage{listings} % Required for insertion of code
\usepackage{courier} % Required for the courier font
\usepackage{lipsum} % Used for inserting dummy 'Lorem ipsum' text into the template
\usepackage{fullpage,amsthm,amsfonts,amssymb,epsfig,amsmath}

% Margins
\topmargin=-0.45in
\evensidemargin=0in
\oddsidemargin=0in
\textwidth=6.5in
\textheight=9.0in
\headsep=0.25in

\linespread{1.1} % Line spacing

% Set up the header and footer
\pagestyle{fancy}
\lhead{\hmwkAuthorName} % Top left header
\chead{\hmwkClass\ (\hmwkClassInstructor\ \hmwkClassTime): \hmwkTitle} % Top center head
\rhead{\firstxmark} % Top right header
\lfoot{\lastxmark} % Bottom left footer
\cfoot{} % Bottom center footer
\rfoot{Page\ \thepage\ of\ \protect\pageref{LastPage}} % Bottom right footer
\renewcommand\headrulewidth{0.4pt} % Size of the header rule
\renewcommand\footrulewidth{0.4pt} % Size of the footer rule
\newcommand{\tab}{\hspace*{3em}}

\setlength\parindent{0pt} % Removes all indentation from paragraphs

%----------------------------------------------------------------------------------------
%   CODE INCLUSION CONFIGURATION
%----------------------------------------------------------------------------------------

\definecolor{MyDarkGreen}{rgb}{0.0,0.4,0.0} % This is the color used for comments
\lstloadlanguages{Perl} % Load Perl syntax for listings, for a list of other languages supported see: ftp://ftp.tex.ac.uk/tex-archive/macros/latex/contrib/listings/listings.pdf
\lstset{language=Perl, % Use Perl in this example
        frame=single, % Single frame around code
        basicstyle=\small\ttfamily, % Use small true type font
        keywordstyle=[1]\color{Blue}\bf, % Perl functions bold and blue
        keywordstyle=[2]\color{Purple}, % Perl function arguments purple
        keywordstyle=[3]\color{Blue}\underbar, % Custom functions underlined and blue
        identifierstyle=, % Nothing special about identifiers                                         
        commentstyle=\usefont{T1}{pcr}{m}{sl}\color{MyDarkGreen}\small, % Comments small dark green courier font
        stringstyle=\color{Purple}, % Strings are purple
        showstringspaces=false, % Don't put marks in string spaces
        tabsize=5, % 5 spaces per tab
        %
        % Put standard Perl functions not included in the default language here
        morekeywords={rand},
        %
        % Put Perl function parameters here
        morekeywords=[2]{on, off, interp},
        %
        % Put user defined functions here
        morekeywords=[3]{test},
        %
        morecomment=[l][\color{Blue}]{...}, % Line continuation (...) like blue comment
        numbers=left, % Line numbers on left
        firstnumber=1, % Line numbers start with line 1
        numberstyle=\tiny\color{Blue}, % Line numbers are blue and small
        stepnumber=5 % Line numbers go in steps of 5
}

% Creates a new command to include a perl script, the first parameter is the filename of the script (without .pl), the second parameter is the caption
\newcommand{\perlscript}[2]{
\begin{itemize}
\item[]\lstinputlisting[caption=#2,label=#1]{#1.pl}
\end{itemize}
}

%----------------------------------------------------------------------------------------
%   DOCUMENT STRUCTURE COMMANDS
%   Skip this unless you know what you're doing
%----------------------------------------------------------------------------------------

% Header and footer for when a page split occurs within a problem environment
\newcommand{\enterProblemHeader}[1]{
\nobreak\extramarks{#1}{#1 continued on next page\ldots}\nobreak
\nobreak\extramarks{#1 (continued)}{#1 continued on next page\ldots}\nobreak
}

% Header and footer for when a page split occurs between problem environments
\newcommand{\exitProblemHeader}[1]{
\nobreak\extramarks{#1 (continued)}{#1 continued on next page\ldots}\nobreak
\nobreak\extramarks{#1}{}\nobreak
}

\setcounter{secnumdepth}{0} % Removes default section numbers
\newcounter{homeworkProblemCounter} % Creates a counter to keep track of the number of problems

\newcommand{\homeworkProblemName}{}
\newenvironment{homeworkProblem}[1][Problem \arabic{homeworkProblemCounter}]{ % Makes a new environment called homeworkProblem which takes 1 argument (custom name) but the default is "Problem #"
\stepcounter{homeworkProblemCounter} % Increase counter for number of problems
\renewcommand{\homeworkProblemName}{#1} % Assign \homeworkProblemName the name of the problem
\section{\homeworkProblemName} % Make a section in the document with the custom problem count
\enterProblemHeader{\homeworkProblemName} % Header and footer within the environment
}{
\exitProblemHeader{\homeworkProblemName} % Header and footer after the environment
}

\newcommand{\problemAnswer}[1]{ % Defines the problem answer command with the content as the only argument
\noindent\framebox[\columnwidth][c]{\begin{minipage}{0.98\columnwidth}#1\end{minipage}} % Makes the box around the problem answer and puts the content inside
}

\newcommand{\homeworkSectionName}{}
\newenvironment{homeworkSection}[1]{ % New environment for sections within homework problems, takes 1 argument - the name of the section
\renewcommand{\homeworkSectionName}{#1} % Assign \homeworkSectionName to the name of the section from the environment argument
\subsection{\homeworkSectionName} % Make a subsection with the custom name of the subsection
\enterProblemHeader{\homeworkProblemName} % Header and footer within the environment
}{
\enterProblemHeader{\homeworkProblemName} % Header and footer after the environment
}

%----------------------------------------------------------------------------------------
%   NAME AND CLASS SECTION
%----------------------------------------------------------------------------------------

\newcommand{\hmwkTitle}{Homework\ \#3} % Assignment title
\newcommand{\hmwkDueDate}{Tuesday,\ April\ 21st,\ 2015} % Due date
\newcommand{\hmwkClass}{CMPS\ 102} % Course/class
\newcommand{\hmwkClassTime}{4:00pm} % Class/lecture time
\newcommand{\hmwkClassInstructor}{Warmuth} % Teacher/lecturer
\newcommand{\hmwkAuthorName}{John Allard \ 1437547
} % Your name


%----------------------------------------------------------------------------------------
%----------------------------------------------------------------------------------------
%   USER SETTINGS
%----------------------------------------------------------------------------------------
%----------------------------------------------------------------------------------------
\usepackage{mathtools}
\DeclarePairedDelimiter\ceil{\lceil}{\rceil}
\DeclarePairedDelimiter\floor{\lfloor}{\rfloor}



%----------------------------------------------------------------------------------------
%   TITLE PAGE
%----------------------------------------------------------------------------------------

\title{
\vspace{2in}
\textmd{\textbf{\hmwkClass:\ \hmwkTitle}}\\
\normalsize\vspace{0.1in}\small{Due\ on\ \hmwkDueDate}\\
\vspace{0.1in}\large{\textit{}}
\vspace{3in}
}

\author{\textbf{\hmwkAuthorName}}
\date{} % Insert date here if you want it to appear below your name

%----------------------------------------------------------------------------------------

\begin{document}

\maketitle

%----------------------------------------------------------------------------------------
%   TABLE OF CONTENTS
%----------------------------------------------------------------------------------------

%\setcounter{tocdepth}{1} % Uncomment this line if you don't want subsections listed in the ToC

% \tableofcontents
\newpage

%----------------------------------------------------------------------------------------
%   PROBLEM 1
%----------------------------------------------------------------------------------------

% To have just one problem per page, simply put a \clearpage after each problem

\begin{homeworkProblem}

\textit{Hadamard matrices} $H_0, H_1, H_2,
\ldots$ are defined as follows:
\begin{itemize} 
\item $H_0$ is the $1 \times 1$ matrix $[1]$
\item For $k>0, H_k$ is the $2^k \times 2^k$ matrix
\end{itemize}
\begin{displaymath}
H_k =
\begin{bmatrix}
H_{k-1} & H_{k-1} \\
H_{k-1} & -H_{k-1} 
\end{bmatrix}
\end{displaymath}

Show by induction that the dot product between any two distinct rows of
such a Hadamard matrix is zero.


\end{homeworkProblem}


%----------------------------------------------------------------------------------------
%   PROBLEM 2
%----------------------------------------------------------------------------------------

\begin{homeworkProblem}

Consider the Coin Changing problem with following set of coins:
\begin{displaymath}
\{ 1, k, k^2, ..., k^n \}.
\end{displaymath}

Prove that the Cashier's Algorithm is optimal for all 
amounts $x \in \mathbb{N}$ given the above set of coins 
for any $k,n\in \mathbb{N}$ s.t. $k\ge2$, $n\geq0$. \\

I will start by showing the maximum number of times that each coin can appear in the optimal solution, then I will use this information to prove that the greedy algorithm will produce the optimal solution. \\

An optimal solution on $k \geq 2$, $n \geq 0$ requires that each coin (except the highest denomination, $c_n = \$k^n)$ is present at most $k-1$ times. Proof : Take a coin $c_m$ of value $k^m$ for $m \geq 2$. If we were to have $k+\alpha$ for $\alpha \geq 0$ $c_m$ coins in our solutions, these coins would account for $(k+\alpha)(k^m) = k^{m+1}+\alpha k$ units of currency. But, (assuming $m+1 \leq n$), we could just replace the $k$ $c_m$ coins with a single $c_{m+1}$ coin, because $k$ $c_m$ coins is $k*k^m = k^{m+1}$ units of currency. Because $k \geq 2$, replacing $k$ coins with a single coin of higher-value will reduce the total number of coins by at least one. Thus any optimal solution, which by definition is the lowest number of coins which add up to the given total, cannot have more than $k-1$ of each coin denomination (except the highest) else we could remove one or more coins and create an even better solution. \\

Now that I have established that each (except the highest value) coin can only appear at most $k-1$ times in our final solution, I will attempt to prove that the cashiers greedy algorithm will find the optimal solution. \\
To start, choose an $m \in \mathbb{N}$ arbitrarily, and let $k \geq 2$ also be chosen arbitrarily. We use a coin-space consisting of $m+1$ different coin values, $\{1, k, k^2, \ldots, k^m \}$, which will form the individual components of our solution. Pick a value $t : 1 \leq t \leq m$, and assume that $\forall y : 1 \leq y < k^t$, that the algorithm produces the optimal solution. Now choose a value of $x : k^t \leq x < k^{t+1}$. In this situation, the greedy algorithm will choose to insert the coin $c_t$ with value $k^t$ into our solution set, because it is the largest value coin that is less than or equal to the value $x$. I wish to show that the optimal solution will also contain this coin. The information revealed in the table below is used to construct this proof. It is inspired from the example in the slides. \\
\begin{tabular}{|c|c|c|c|} \hline
$p$ & $k^p$ & restrictions   & max val of $c_0, c_1, \ldots, c_{p-1}$              \\ \hline
0     & 1     & $c_o \leq k-1$ & /                                                 \\
1     & $k$   & $c_1 \leq k-1$ & $1*(k-1)$                                         \\
2     & $k^2$ & $c_2 \leq k-1$ & $1*(k-1) + k*(k-1) = k^2-1$                       \\
3     & $k^3$ & $c_3 \leq k-1$ & $1*(k-1) + k*(k-1) + k^2*(k-1) = k^3-1$           \\
      &       &    ....        &          ...                                      \\
n     & $k^n$ & none           & $(k-1)*\sum_{i=0}^{n-1} k^i$                        \\ \hline
\end{tabular}

I will concentrate on the general term. For any given coin $c_n$, the maximum combined values of all coins less than that coin, $M_{n}$ is defined as :
$$ M_{n} = (k-1)*\sum_{i=0}^{n} k^i $$
This is of course a geometric series with common ratio $k$, which reduces to the following equation.
$$ M_{n} = (k-1)*\frac{(1-k^{n})}{1-k} $$
$$ M_{n} = \frac{(k-1)}{(1-k)}*1-k^{n} $$
$$ M_{n} = -1*(1-k^{n}) $$
$$ M_{n} = k^{n} - 1 $$

Thus, we claimed that given a value $x : k^t \leq x < k^{t+1}$, the greedy algorithm will select the coin $c_t$ with a value of $k_t$ and that any optimal solution will also contain this coin. For this to be true, there must be no other way to creating a value of $x$ with coins that are of a value less than $k^t$. The proof above showed that given a coin $c_n$, the max combined value ($M_n$) of all coins less than that coin is given by the equation $ M_{n} = k^{n} - 1$. This is strictly less than $k^n$, which is the value of the coin $c_n$. Thus there is no valid way to construct the value $x$ by selecting a coin other than $c_t$, which means the optimal solution will also contain $c_t$.  \\

We now recurse on a problem of size $x - c_t$, if this value is still greater than $c_t$, the greedy algorithm will select another $c_t$ coin, and the argument above again applies, confirming that the optimal solution will also contain another $c_t$ coin. This process can repeat at most $k-1$ times (we can have at most that many of each coin), before we will be at a value $x < k^t$, so the inductive hypothesis applies and we infer that the optimal solution will be produced by our greedy algorithm. \\

\end{homeworkProblem}

\begin{homeworkProblem}
You are given an unbounded array $A[1],A[2],A[3], \ldots]$ containing
distinct integers sorted in ascending order. Describe an
efficient algorithm that takes an integer $k$ as input and
finds out whether $k$ is in the array in time
$O(\log p)$ time where $p$ is the number of integers in the
array that are strictly less than $k$.
\end{homeworkProblem}


\begin{homeworkProblem}

We define an array $A$ of $n$ objects has a \emph{dominant} object if at least $\lfloor n/2 \rfloor+1$ entries of $A$ are identical. Our goal is to design an efficient algorithm to tell whether the array has a dominant object, and, if so, to find that object. Our only access to $A$ is by making a \textit{query} asking whether $A[i] = A[j]$ for any two $i,j \in \{ 1, 2, \ldots, n \}$. 
\begin{enumerate}
\item[\textbf{(a)}] Design an algorithm to solve this problem with $O(n \log n)$ queries. (\textbf{Hint:} Split the array $A$ into two arrays of half the size.)
\item[\textbf{(b)}] Design an algorithm to solve this problem with $O(n)$ queries. (\textbf{Hint:} Don't Split. Pair up the elements arbitrarily and get rid of as many as you can, repeatedly.)
\end{enumerate}

\end{homeworkProblem}



\begin{homeworkProblem}

KT, problem 17, p 197.)\\
Consider the following variation on the Interval Scheduling Problem.You have a processor that can operate 24 hours a day, every day. People submit requests to run \textit{daily jobs} on the processor. Each such job comes with a \textit{start time} and an \textit{end time}; if the job is accepted to run on the processor, it must run continuously, every day, for the period between its start and end times. (Note that certain jobs can begin before midnight and end after midnight; this makes for a type of situation different from what we saw in the Interval Scheduling Problem.)\\

Given a list of $n$ such jobs, your goal is to accept as many jobs as possible (regardless of their length), subject to the constraint that the processor can run at most one job at any given point in time. Provide an algorithm to do this with a running time that is polynomial in $n$. You may assume for simplicity that no two jobs have the same start or end times.\\

\textbf{Example.} Consider the following four jobs, specified by (\textit{start-time, end- time}) pairs.\\
\begin{displaymath}
\text{(6 PM, 6 AM), (9 PM, 4 AM), (3 AM, 2 PM), (1 PM, 7 PM)}
\end{displaymath}

The optimal solution would be to pick the two jobs (9 P.M., 4 A.M.) and (1 P.M., 7 P.M.), which can be scheduled without overlapping.

\end{homeworkProblem}



\begin{homeworkProblem}

(KT, problem 4, p 190)\\
Sorry too long to retype. Use a greedy algorithm.
Reason your time bound as well as correctness.

\end{homeworkProblem}


\begin{homeworkProblem}

EC: We discussed an algorithm in class
that finds the $k$-th largest element in an array for $n\ge
k$ elements in worst case time $O(n)$. This algorithm
starts by finding the medians of groups of 5 elements and
then finds the medians of the roughly $n/5$ medians.

Why is this algorithm based on groups of size 5?
Does it also work with groups of size 3? Why or why not?

\end{homeworkProblem}

%----------------------------------------------------------------------------------------

\end{document}