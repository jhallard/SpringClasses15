\documentclass[11pt,dvipsnames]{article}

\usepackage{fullpage,amssymb,amsmath,amsthm}
\usepackage{color}
\usepackage{url}
\usepackage[all]{xy}
\usepackage{algorithm}
\usepackage{algorithmic}
\usepackage{subfigure}
\usepackage{booktabs}
\usepackage{graphicx}

\usepackage{tikz}
\usepackage{verbatim,array}
\usepackage[active,tightpage]{preview}
\usetikzlibrary{decorations.pathmorphing,shapes,positioning,calc,fit,arrows,arrows.meta}
\setlength\PreviewBorder{5pt}

\usepackage{scrextend}
\newcommand{\tikzmark}[1]{%
	\tikz[overlay,remember picture] \node(#1){};
}

\title{CMPS102: Homework \#7}
\author{Sebastien Young}
\date{May 26, 2015}

\newcommand{\KT}[2]{Kleinberg \& Tardos, Ch. #1, \##2}
\newcommand{\TODO}[1]{{\color{red} \Huge \textbf{TODO:} #1}}
\newcommand{\alg}[1]{{\sc #1}}

\theoremstyle{remark}
\newtheorem*{claim}{Claim}

\setcounter{secnumdepth}{1}

\begin{document}
\maketitle

\begin{enumerate}
\item Consider the longest common subsequence problem
between two strings $x_1,...,x_n$ and $y_1,...,y_m$.
Define a graph over the $n\times m$ grid (plus possibly some
vertices around the edges) s.t. the longest common
subsequence corresponds to the longest path in this graph.
\begin{itemize}
	\renewcommand\labelitemi{-}
	\item Clearly describe the condition for the presence of an
  edge between two vertices on the grid.
	\item How should the edges be labeled?
	\item How do you find the longest path?
	\item Is this algorithm more efficient than the dynamic
  programming algorithm
\end{itemize}

%%%%%%%%%%%%%%%%%%%%%%%%%%%%%%%%%%%%%%%%%%%%%%%%%%%%%%%%%%%
%%%%%%%%%%%%%%%%%%%%%%%%%%%%%%%%%%%%%%%%%%%%%%%%%%%%%%%%%%%
%%%%%%%%%%%%%%%%%%%%%%%%%%%%%%%%%%%%%%%%%%%%%%%%%%%%%%%%%%%
\newpage
\item \KT{7}{3}
\begin{center}
\begin{tikzpicture}
	\node[draw,circle] (s) {$s$};
	\node[draw,circle] (b) [right=2cm of s] {$b$};
	\node[draw,circle] (c) [right=2cm of b] {$c$};
	\node[draw,circle] (t) [right=2cm of c] {$t$};
	\node (m) at ($(b)!0.5!(c)$) {};
	\node[draw,circle] (a) [above=1.73cm of m] {$a$};
	\node[draw,circle] (d) [below=1.73cm of m] {$d$};
	\draw[-latex] (s) -- (a) node [pos=.6, above left,draw] {6} node [pos=.4,above] {10};
	\draw[-latex] (s) -- (b) node [pos=.5, above] {3} node [pos=.5, below, draw] {3};
	\draw[-latex] (s) -- (d) node [pos=.4, below] {1} node [pos=.6, below left, draw] {1};
	\draw[-latex] (a) -- (b) node [pos=.6, below] {1} node [pos=.4,below right, draw] {1};
	\draw[-latex] (a) -- (c) node [pos=.5, right]{2};
	\draw[-latex] (a) -- (t) node [pos=.4,above]{5} node [pos=.6,above right, draw]{5};
	\draw[-latex] (b) -- (c) node [pos=.5,above]{6} node [pos=.5,below,draw] {5};
	\draw[-latex] (c) -- (t) node [pos=.5,above]{5} node[pos=.5,below,draw]{5};
	\draw[-latex] (d) -- (b) node [pos=.6,below left,draw] {1} node [pos=.5,right] {3};
	\draw[-latex] (d) -- (c) node [pos=.5, right]{3};
	\draw[-latex] (d) -- (t) node [pos=.5,below]{10};
\end{tikzpicture}
\end{center}
\begin{enumerate}
	\item What is the value of this flow?
Is this a maximum $(s,t)$ flow in this graph?
	\item Find a minimum $s-t$ cut in the flow network pictured 
in Figure 7.27[above], and also say what its capacity is.
\end{enumerate}

%%%%%%%%%%%%%%%%%%%%%%%%%%%%%%%%%%%%%%%%%%%%%%%%%%%%%%%%%%%
%%%%%%%%%%%%%%%%%%%%%%%%%%%%%%%%%%%%%%%%%%%%%%%%%%%%%%%%%%%
%%%%%%%%%%%%%%%%%%%%%%%%%%%%%%%%%%%%%%%%%%%%%%%%%%%%%%%%%%%
\newpage
\item \KT{7}{4}\\
Decide whether you think the following statement is 
true, give a short explanation. If it is false, give a counterexample.
\begin{addmargin}[2em]{0em}
\textit{%
Let G be an arbitrary flow network, with a source s, a sink t, and a positive
integer capacity $c_e$ on every edge e. If f is a maximum s-t flow in G, then f
saturates every edge out of s with flow (i.e., for all edges e out of s, we have
$f(e)=c_e$).}
\end{addmargin}

%%%%%%%%%%%%%%%%%%%%%%%%%%%%%%%%%%%%%%%%%%%%%%%%%%%%%%%%%%%
%%%%%%%%%%%%%%%%%%%%%%%%%%%%%%%%%%%%%%%%%%%%%%%%%%%%%%%%%%%
%%%%%%%%%%%%%%%%%%%%%%%%%%%%%%%%%%%%%%%%%%%%%%%%%%%%%%%%%%%
\newpage
\item \KT{7}{5}\\
Decide whether you think the following statement is true or false. If it is
true, give a short explanation. If it is false, give a counterexample.
\begin{addmargin}[2em]{0em}
\textit{%
Let G be an arbitrary flow network, with a source s, a sink t, and a positive
integer capacity $c_e$ on every edge e; and let $(A,B)$ be a minimum s-t cut with
respect to these capacities $\{c_e:e\in E\}$. Now suppose we add $1$ to every capacity;
then $(A,B)$ is still a minimum s-t cut with respect to these new capacities
$\{1+c_e:e\in E\}$.
}
\end{addmargin}

%%%%%%%%%%%%%%%%%%%%%%%%%%%%%%%%%%%%%%%%%%%%%%%%%%%%%%%%%%%
%%%%%%%%%%%%%%%%%%%%%%%%%%%%%%%%%%%%%%%%%%%%%%%%%%%%%%%%%%%
%%%%%%%%%%%%%%%%%%%%%%%%%%%%%%%%%%%%%%%%%%%%%%%%%%%%%%%%%%%
\newpage
\item \KT{7}{10}\\
Suppose you are given a direced graph $G=(V,E)$, with a positive integer
capacity $c_e$ on each edge $e$,...
a maximum $s-t$ flow in $G$, defined by a flow value $f_e$ on each edge
\par Now, suppose we pick a specific edge $e^*\in E$ and reduce its capacity
by 1 unit. Show how to find a maximum flow in the resulting capacitated
graph in time $\mathcal{O}(m+n)$, where $m$ is the number of edges in $G$ and $n$ is the
number of nodes.

%%%%%%%%%%%%%%%%%%%%%%%%%%%%%%%%%%%%%%%%%%%%%%%%%%%%%%%%%%%
%%%%%%%%%%%%%%%%%%%%%%%%%%%%%%%%%%%%%%%%%%%%%%%%%%%%%%%%%%%
%%%%%%%%%%%%%%%%%%%%%%%%%%%%%%%%%%%%%%%%%%%%%%%%%%%%%%%%%%%
\newpage
\item CRLS, Ch. 26.1, exercise 6\\
Professor Adam has two children who, unfortunately, dislike each other. The problem
is so severe that not only do they refuse to walk to school together, but in fact
each one refuses to walk on any block that the other child has stepped on that day.
The children have no problem with their paths crossing at a corner. Fortunately
both the professor's house and the school are on corners, but beyond that he is not
sure if it is going to be possible to send both of his children to the same school.
The professor has a map of his town. Show how to formulate the problem of determining
whether both his children can go to the same school as maximum-flow problem.
\end{enumerate}
\end{document}